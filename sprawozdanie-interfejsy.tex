\documentclass[a4paper,11pt,notitlepage]{article}
\usepackage{sprawozdanie-ato}

\begin{document}


\title{\
Laboratorium Sieci Komputerowych\\\
Konfiguracja łącz i interfejsów sieciowych\
}
\author{\
Tomasz Cudziło, Barnaba Turek\\
\textsc{PW EE Informatyka}\\[6pt]
}
\date{\today}

\maketitle
\tableofcontents


\section{Cel ćwiczenia}

W ramach laboratorium mieliśmy za zadanie własnoręcznie skonfigurować i
przetestować kilka interfejsów sieciowych w systemie \bsd. Spośród dostępnych
łącz postanowiliśmy przetestować:

\begin{description}
    \item[łącze przewodowe \eth\textnormal{,}] które było dostępne na naszej stacji roboczej
        i podłączone fizycznie do sieci laboratorium, jednak nie było
        skonfigurowane na maszynie.
    \item[łącze radiowe \wifi\textnormal{,}] w ramach testowania którego, wypróbowaliśmy
        dwa tryby pracy: połączenie typu \emph{punkt-punkt (P2P)}, oraz
        połączenie do \emph{punktu dostępowego (AP)}.
    \item[łącze radiowe \bt\textnormal{,}] dla którego połączyliśmy dwa stanowiska,
        analogicznie do połączenia typu \wifi{} \emph{P2P}.
    \item[łącze szeregowe \uart\textnormal{,}] poprzez które połączyliśmy konsolą maszyny
        \zielone{} za pomocą kabla.
\end{description}


\section{Istniejąca konfiguracja stanowiska}


\section{Wykonanie ćwiczenia}


\section{Wnioski i spostrzeżenia}


\end{document}
