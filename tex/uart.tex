\section{Połączenie szeregowe}
\subsection{Wprowadzenie}
Systemy z~rodziny *NIX pozwalają na korzystanie z~konsoli za pośrednictwem portu szeregowego (zwykle rs\dywiz 232).
Funkcja ta jest przydatna zarówno dla administratorów (ponieważ nie trzeba podłączać klawiatury i~monitora do serwera) jak i~programistów, pracujących nad sterownikami.

Skorzystaliśmy z~konsoli aby połączyć się z~urządzeniem \zielone{}\todo{jak się nazywało to zielone?}.

Do zestawienia takiego połączenia za pomocą portu szeregowego potrzebny jest kabel typu \emph{null\dywiz modem}\cite{serial-console}.

\subsection{Wykonanie Ćwiczenia - null-modem}
\label{sec:serial:null-modem}

Po załadowaniu niezbędnych sterowników i~podłączeniu kabla \emph{null\dywiz modem} zajrzeliśmy do pliku \texttt{/etc/remote}, w~którym zdefiniowane są systemy znane przez program \texttt{tip}.

Dla każdego systemu podana jest nazwa urządzenia oraz konfiguracja połączenia (ustawienia takie jak typ kontroli parzystości i~prędkość transmisji).

\begin{lstlisting}[caption={\texttt{/etc/remote}}]
#ATO:
modem:dv=/dev/cuau0:br#115200:pa=none:
ps:dv=/dev/cuau0:br#115200:pa=none:
bt:dv=/dev/ttyp0:br#115200:pa=none:
\end{lstlisting}

Skorzystaliśmy z~urządzenia \texttt{ps}, ponieważ wynik polecenia \texttt{sterowniki -s} sugerował, że mamy się połączyć z urządzeniem \texttt{/dev/cuau0}.
Do tego prędkość transmisji zgadzała się z zaproponowanymi w wyjściu polecenia.

\begin{lstlisting}[caption={Połączenie z konsolą za pomocą programu tip}]
k9% sudo tip ps
^Gconnected

Login incorrect
debian login:
debian login: +++RESET

POST: 012345689bcefghips1234ajklnopqr,,,tvwxy

comBIOS ver. 1.33c 20080626  Copyright (C) 2000-2008 Soekris Engineering.

\end{lstlisting}

Zastosowaliśmy polecenie z zestawu rozkazów Hayesa żeby zrestartować maszynę \zielone{} i z pomocą prowadzącego uruchomiliśmy na niej system FreeBSD.

\subsection{Wykonanie Ćwiczenia - bluetooth}
Dzięki zastosowaniu modemu radiowego \emph{firefly} możliwe jest bezprzewodowe połączenie z~konsolą.

Modem Firefly posiada tryb pracy, w~którym w~przezroczysty sposób zastępuje kabel szeregowy, jednak wymagane są dwa modemy na każde połączenie.

Zamiast tego w~laboratorium każda z~maszyn \zielone{} ma podpiety jeden modem, a~komputery - terminale - łączą się z tym modemem za pomocą własnego, niespecjalizowanego radia Bluetooth.

Rozwiązanie takie pozwala ograniczyć liczbę potrzebnych modemów \emph{Firefly}, wymaga jednak więcej konfiguracji.

Po załadowaniu niezbędnych sterowników i~wyszukaniu interesujących nas urządzeń w~eterze (analogicznie do ćwiczenia \ref{sec:bt}), próbowaliśmy skorzystać ze skryptu \texttt{bt\dywiz spp}.
Niestety skrypt aktualnie nie działa.

Wykryliśmy, że problem dotyczy tworzenia nowego terminala przez polecenie \texttt{rfcomm\_sppd}.
Wyłączenie tej funkcjonalności ograniczyłoby zastosowania skryptu \texttt{bt\dywiz spp}, ale na nasze potrzeby było wystarczające.
Po wyłączeniu tworzenia urządzenia - pseudo\dywiz terminala konsola, z~którą chcemy się połaczyć jest dostępna w~konsoli, z~której się łaczyliśmy (analogicznie do działania programu \texttt{tip} w~sekcji \ref{sec:serial:null-modem}).

\begin{lstlisting}[caption={Połączenie z konsolą za pomocą programu \texttt{rfcomm\_sppd}}]
 $ hostname
k9
 $ rfcomm_sppd -a F11
rfcomm_sppd[1656]: Starting on stdin/stdout...
Login:stud
Password:zetis

Last login: Fri Apr 13 04:57:21 on ttyu0
FreeBSD 9.0-STABLE (SOEKRIS) #0 r233366M: Sat Mar 24 00:57:52 CET 2012

so4801% hostname
hostname
so4801
\end{lstlisting}
