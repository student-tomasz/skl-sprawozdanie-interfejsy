\subsection{Połączenie szeregowe}

\subsubsection{Wprowadzenie}

Systemy z~rodziny *NIX pozwalają na korzystanie z~konsoli za pośrednictwem portu
szeregowego (zwykle rs\dywiz 232). Funkcja ta jest przydatna zarówno dla
administratorów (ponieważ nie trzeba podłączać klawiatury i~monitora do serwera)
jak i~programistów, pracujących nad sterownikami.

Skorzystaliśmy z~konsoli aby połączyć się z~urządzeniem \zielone{} z maszyny
\kdziew. Do zestawienia takiego połączenia za pomocą portu szeregowego potrzebny
jest kabel typu \emph{null\dywiz modem}\cite{serial-console}.

\subsubsection{Konfiguracja połączenia --- null\dywiz modem}
\label{sec:serial:null-modem}

Po załadowaniu niezbędnych sterowników i~podłączeniu kabla \emph{null\dywiz
modem} zajrzeliśmy do pliku \texttt{/etc/remote}, w~którym zdefiniowane są
systemy rozpoznawalne przez program \texttt{tip}. Dla każdego systemu podana
jest nazwa urządzenia oraz konfiguracja połączenia, gdzie ustalane są m.in. typ
kontroli parzystości i~prędkość transmisji.

\begin{lstlisting}[caption={Linie z \texttt{/etc/remote} dotyczące naszego połączenia.}]
modem:dv=/dev/cuau0:br#115200:pa=none:
ps:dv=/dev/cuau0:br#115200:pa=none:
bt:dv=/dev/ttyp0:br#115200:pa=none:
\end{lstlisting}

Skorzystaliśmy z~urządzenia \texttt{ps}, ponieważ instrukcja ze skryptu
\texttt{sterowniki~-s} sugerowała połączenie wykorzystujące urządzenie
\texttt{/dev/cuau0}. Dodatkowo zadeklarowana w \texttt{/etc/remote} prędkość
transmisji zgadzała się z prędkością zaproponowaną w instrukcji.

\begin{lstlisting}[caption={Połączenie z konsolą maszyny \zielone{} za pomocą programu \texttt{tip}.}]
k9% sudo tip ps
^Gconnected

Login incorrect
debian login:
debian login: +++RESET

POST: 012345689bcefghips1234ajklnopqr,,,tvwxy

comBIOS ver. 1.33c 20080626  Copyright (C) 2000-2008 Soekris Engineering.
\end{lstlisting}

Żeby zrestartować maszynę \zielone{} zastosowaliśmy polecenie z zestawu rozkazów
Hayesa tj. wpisaliśmy \texttt{+++} i po odczekaniu chwili wydaliśmy polecenie
\texttt{RESET}. Następnie z pomocą prowadzącego uruchomiliśmy na \zielone{}
system \bsd.

\subsubsection{Konfiguracja połączenia --- \bt}

Dzięki zastosowaniu modemu radiowego Firefly możliwe jest bezprzewodowe
połączenie z~konsolą. Modem Firefly posiada tryb pracy, w~którym w~przezroczysty
sposób zastępuje kabel szeregowy, jednak wymagane są dwa modemy na każde
połączenie.

Zamiast tego w~laboratorium każda z~maszyn Soekris ma podpięty jeden modem,
a~komputery\dywiz terminale łączą się z tym modemem za pomocą własnego,
niespecjalizowanego radia \bt. Rozwiązanie takie pozwala ograniczyć liczbę
potrzebnych modemów Firefly, wymaga jednak więcej konfiguracji.

Po załadowaniu niezbędnych sterowników i~wyszukaniu interesujących nas urządzeń
w~eterze (analogicznie do ćwiczenia \ref{sec:bt}), próbowaliśmy skorzystać ze
skryptu \texttt{bt-spp}. Niestety skrypt aktualnie nie daje oczekiwanych
rezultatów.

Wykryliśmy, że problem dotyczy tworzenia nowego terminala przez polecenie
\texttt{rfcomm\_sppd}. Wyłączenie tej funkcjonalności ograniczyłoby zastosowania
skryptu \texttt{bt-spp}, ale na nasze potrzeby było wystarczające. Usunęliśmy ze
skryptu komendę tworzącą urządzenie pełniące rolę pseudo\dywiz terminala. W
efekcie konsola, z~którą chcemy się połączyć jest dostępna w~konsoli, z~której
się łączyliśmy, analogicznie do działania programu \texttt{tip} w~sekcji
\ref{sec:serial:null-modem}.

\begin{lstlisting}[caption={Połączenie z konsolą maszyny Soekris za pomocą programu \texttt{rfcomm\_sppd}.}]
# Na maszynie k9.
k9% hostname
k9

# Laczymy sie z maszyna Soekris poprzez modem Firefly.
k9% rfcomm_sppd -a F11
rfcomm_sppd[1656]: Starting on stdin/stdout...
Login:stud
Password:zetis

Last login: Fri Apr 13 04:57:21 on ttyu0
FreeBSD 9.0-STABLE (SOEKRIS) #0 r233366M: Sat Mar 24 00:57:52 CET 2012

# Jestesmy w konsoli maszyny Soekris.
so4801% hostname
so4801
\end{lstlisting}
