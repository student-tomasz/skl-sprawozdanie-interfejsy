\subsection{\wifi}

\subsubsection{Wprowadzenie}

Pod terminem \wifi{} rozumiana jest sieć WLAN --- Wireless Local Area Network w
standardzie \texttt{IEEE 802.11}. W laboratorium urządzenia pracują zgodnie z
wersją \texttt{IEEE 802.11g} standardu. Standard zakłada pracę na paśmie
częstotliwości 2400 MHz \cite{wifi:pasmo} i oferuje szybkość do 54 Mbps
\cite{wifi:szybkosc} w zasięgu do 38 metrów od punktu dostępowego. W paśmie
standardu \texttt{g} mieści się 14 kanałów. W Polsce można używać kanały od 1.
do 13., czyli pasmo 2400 MHz do 2483,5 MHz. Przeczytano wstęp obsługi sieci
bezprzewodowych \cite{wifi:bsd-handbook} dla systemu \bsd{} i przystąpiono do
wykonania ćwiczenia.

Do maszyny \kdziew{} podłączone jest urządzenie \wifi{} przez port USB. By móc z
niego skorzystać postąpiono zgodnie z instrukcjami ze skryptu \texttt{sterowniki
-w}.

\begin{lstlisting}
k9% sudo kldload uhci ehci
k9% lsusb
[...]
ugen4.2: <Wireless-G ProtableUSB Adapter Cisco-Linksys> at usbus4, cfg=0 md=HOST spd=HIGH (480Mbps) pwr=ON
\end{lstlisting}

Po załadowaniu sterowników USB, urządzenie \wifi{} zostało wykryte przez system.
Załadowano sterowniki dla samego urządzenia \wifi{} i stwierdzono powstanie
nowego interfejsu o nazwie \texttt{ural0}.

\begin{lstlisting}
k9% sudo kldload if_ural
k9% ifconfig
[...]
ural0: flags=8802<BROADCAST,SIMPLEX,MULTICAST> metric 0 mtu 2290
    ether 00:12:17:60:ee:f8
    media: IEEE 802.11 Wireless Ethernet autoselect <adhoc> (autoselect <adhoc>)
    status: no carrier
\end{lstlisting}

Z tak załadowanym radiem \wifi{}, zaczęto konfigurować połączenie do sieci WLAN
\texttt{ZETiIS}.

\subsubsection{Konfiguracja połączenia --- do punktu dostępowego}

Stworzono pseudo\dywiz interfejs \texttt{wlan0}, który korzysta z rzeczywistego
interfejsu \texttt{ural0}. Nie podano specjalnego trybu pracy urządzenia,
zostaje wybrany tryb domyślny. Ustawiono polskie restrykcje dotyczące dostępnego
pasma i sprawdzono stan nowego interfejsu.

\begin{lstlisting}
k9% sudo ifconfig wlan0 create wlandev ural0
k9% sudo ifconfig wlan0 country PL
k9% ifconfig wlan0
wlan0: flags=8802<BROADCAST,SIMPLEX,MULTICAST> metric 0 mtu 1500
    ether 00:12:17:60:ee:f8
    media: IEEE 802.11 Wireless Ethernet autoselect (autoselect)
    status: no carrier
    ssid "" channel 1 (2412 MHz 11b)
    regdomain ETSI country PL authmode OPEN privacy OFF txpower 30
    bmiss 7 scanvalid 60 bgscan bgscanintvl 300 bgscanidle 250 roam:rssi 7
    roam:rate 1 bintval 0
\end{lstlisting}

Uruchomiono interfejs \texttt{wlan0} oraz przeszukano dostępne sieci \wifi.

\begin{lstlisting}
k9% sudo ifconfig wlan0 up
k9% sudo ifconfig wlan0 scan
SSID/MESH ID    BSSID              CHAN RATE   S:N     INT CAPS
[...]
ZETiIS          00:1e:52:79:ca:59    6   54M -74:-95  100 EPS  RSN HTCAP WME
pwwifi-stud...  00:24:14:31:95:03   11   54M -87:-95  102 ES   HTCAP WME
pwwifi-stud...  00:21:a0:81:86:c3    1   54M -83:-95  102 ES   HTCAP WME
pwwifi          00:24:14:31:95:00   11   54M -87:-95  102 ES   HTCAP WME
pwwifi          00:21:a0:0f:e2:40   11   54M -88:-95  102 ES   HTCAP WME
\end{lstlisting}

Razem z listą innych sieci, widoczna jest sieć \texttt{ZETiIS} z atrybutami
\texttt{EPS}. Każdy ze znaków informuje o danej cesze sieci. Interesujące nas
cechy to:

\begin{description}
    \item[E] --- aktualny nadajnik jest częścią większej infrastruktury, nie
        pracuje w trybie ad\dywiz hoc,
    \item[P] --- sieć wymaga autoryzacji użytkownika. Dla porównania widać, że
        sieć \texttt{pwwifi} nie jest zabezpieczona na tej warstwie.
\end{description}

Podano jawną nazwę sieci, do której chcemy się połączyć. Nie uzyskano
połączenia. Niezbędny jest sekret uprawniający do skorzystania z sieci.

\begin{lstlisting}
k9% sudo ifconfig wlan0 ssid ZETiIS
k9% sudo ifconfig wlan0
wlan0: flags=8843<UP,BROADCAST,RUNNING,SIMPLEX,MULTICAST> metric 0 mtu 1500
    ether 00:12:17:60:ee:f8
    media: IEEE 802.11 Wireless Ethernet autoselect (autoselect)
    status: no carrier
    ssid ZETiIS channel 11 (2462 MHz 11g)
    regdomain ETSI country PL authmode WPA2/802.11i privacy OFF
    txpower 30 bmiss 7 scanvalid 60 bgscan bgscanintvl 300 bgscanidle 250
    roam:rssi 7 roam:rate 5 protmode CTS bintval 102
\end{lstlisting}

Ponieważ sieć \texttt{ZETiIS} korzysta z szyfrowania WPA w wersji korporacyjnej,
potrzebne są dane autoryzujące użytkownika, oraz wymiana kluczy i szyfrowanie,
czym zajmuje się usługa \texttt{wpa\_supplicant}. Włączono demona
\texttt{wpa\_supplicant} dla interfejsu \texttt{wlan0}, podano sekret
użytkownika i nawiązano połączenie z siecią.

\begin{lstlisting}
k9% sudo service wpa_supplicant start wlan0
Starting wpa_supplicant.

k9% sudo wpa_cli ping
PONG

k9% sudo wpa_cli status
wpa_state=DISCONNECTED
Supplicant PAE state=DISCONNECTED
suppPortStatus=Unauthorized
EAP state=DISABLED
selectedMethod=25 (EAP-PEAP)
EAP TLS cipher=DHE-RSA-AES256-SHA
EAP-PEAPv0 Phase2 method=MSCHAPV2

k9% sudo wpa_cli identity ZETiIS cudzilot
OK

k9% sudo wpa_cli password ZETiIS haslo12345
OK

k9% sudo wpa_cli select_network ZETiIS
OK

k9% sudo wpa_cli status
bssid=00:1e:52:79:ca:59
ssid=ZETiIS
id=0
mode=station
pairwise_cipher=CCMP
group_cipher=CCMP
key_mgmt=WPA2/IEEE 802.1X/EAP
wpa_state=COMPLETED
Supplicant PAE state=AUTHENTICATED
suppPortStatus=Authorized
EAP state=SUCCESS
selectedMethod=25 (EAP-PEAP)
EAP TLS cipher=DHE-RSA-AES256-SHA
EAP-PEAPv0 Phase2 method=MSCHAPV2

k9% sudo ifconfig wlan0
wlan0: flags=8843<UP,BROADCAST,RUNNING,SIMPLEX,MULTICAST> metric 0 mtu 1500
    ether 00:12:17:60:ee:f8
    media: IEEE 802.11 Wireless Ethernet DS/1Mbps mode 11g
    status: associated
    ssid ZETiIS channel 6 (2437 MHz 11g) bssid 00:1e:52:79:ca:59
    regdomain ETSI country PL authmode WPA2/802.11i privacy ON
    deftxkey UNDEF AES-CCM 3:128-bit txpower 30 bmiss 7 scanvalid 450
    bgscan bgscanintvl 300 bgscanidle 250 roam:rssi 7 roam:rate 5
    protmode CTS roaming MANUAL
\end{lstlisting}

Następnie skonfigurowano warstwę 3. dla interfejsu \texttt{wlan0}, za pomocą
DHCP. Analogicznie do konfiguracji z sekcji \ref{sec:eth:konfiguracja}. Łącze
przetestowano wysyłając zapytania z \volt{} i nasłuchując ruch na interfejsie.

\begin{lstlisting}
k9% sudo dhclient wlan0
DHCPDISCOVER on wlan0 to 255.255.255.255 port 67 interval 4
DHCPDISCOVER on wlan0 to 255.255.255.255 port 67 interval 9
DHCPOFFER from 10.146.7.3
unknown dhcp option value 0xaf
DHCPREQUEST on wlan0 to 255.255.255.255 port 67
DHCPACK from 10.146.7.3
unknown dhcp option value 0xaf
bound to 10.146.226.113 -- renewal in 1800 seconds.

volt% ping -c 1 10.146.226.113
PING 10.146.226.113 (10.146.226.113): 56 data bytes
64 bytes from 10.146.226.113: icmp_seq=0 ttl=64 time=2.822 ms
--- 10.146.226.113 ping statistics ---
1 packets transmitted, 1 packets received, 0.0% packet loss
round-trip min/avg/max/stddev = 2.822/2.822/2.822/0.000 ms

k9% sudo tcpdump -i wlan0 icmp
tcpdump: verbose output suppressed, use -v or -vv for full protocol decode
listening on wlan0, link-type EN10MB (Ethernet), capture size 65535 bytes
00:35:53.738249 IP 10.146.7.3 > 10.146.226.113: ICMP echo request, id 22633, seq 0, length 64
00:35:53.738287 IP 10.146.226.113 > 10.146.7.3: ICMP echo reply, id 22633, seq 0, length 64
^C
2 packets captured
259 packets received by filter
0 packets dropped by kernel
\end{lstlisting}

Na koniec ćwiczenia zatrzymano demon \texttt{wpa\_supplicant} i usunięto
pseudo\dywiz interfejs \texttt{wlan0}.

\begin{lstlisting}
k9% sudo service wpa_supplicant stop wlan0
Stopping wpa_supplicant.
Waiting for PIDS: 1275.
k9% sudo ifconfig wlan0 destroy
\end{lstlisting}

%\subsubsection{Konfiguracja połączenia --- tryb ad\dywiz hoc}
