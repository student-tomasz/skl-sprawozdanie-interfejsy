\documentclass[a4paper,11pt,notitlepage]{article}
\usepackage[utf8]{inputenc}	% latin2 - kodowanie iso-8859-2; cp1250 - kodowanie windows
\usepackage[T1]{fontenc}
\usepackage[polish]{babel}
\usepackage[MeX]{polski}
\selectlanguage{polish}

\usepackage{graphicx}

\hyphenation{FreeBSD}

\author{Jan Kowalski gr.3}
\title{Laboratorium Sieci Komputerowych - 1 \\ {\small Konfiguracja łącz i interfejsów sieciowych}}
\date{\today}

\linespread{1.3}

\usepackage{indentfirst}

\begin{document}
\maketitle
\tableofcontents

\section{Przykładowy rozdział}
Do edycji tego dokumentu został użyty program ``kile''. W systemie windows można użyć edytora ``WinEdt''.

\subsection{Przykładowy podrozdział}
Wyświetlenie plików z maską przy pomocy polecenie \verb+ls+.
\footnotesize\begin{verbatim}
% ls -l /etc/*passwd
-rw-------  1 root  wheel  171843 23 lut 12:39 /etc/master.passwd
-rw-r--r--  1 root  wheel  120984 23 lut 12:39 /etc/passwd
\end{verbatim}

\section{Wnioski}


\end{document}
